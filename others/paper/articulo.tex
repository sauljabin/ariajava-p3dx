%%%%%%%%%%%%%%%%%%%%%%%%%%%%%%%%%%%%%%%%%%%%%%%%%%%%%%%%%
%Este documento representa la plantilla de los articulos%
%a ser editados para la revista, realizada bajo codigo  %
%        Latex, con una clase ``article``               %   
%          PUBLICACIONES EN CIENCIAS                    %               
%                Y  TECNOLOGIA.                         %
%        Realizado por Adriana Araujo                   %
%       Revisado por Hugo Lara              (2014)      %        
%              Cuerpo editorial.                        %
%%%%%%%%%%%%%%%%%%%%%%%%%%%%%%%%%%%%%%%%%%%%%%%%%%%%%%%%%
\documentclass[11pt,twoside,A5]{article}
\usepackage[spanish]{babel}
\usepackage[utf8]{inputenc}
\usepackage{amsmath}
\usepackage{amsfonts}
\usepackage{amssymb}
\usepackage{amscd}
\usepackage{psfrag}
\usepackage{graphicx}
\usepackage{url}
%%%%%%%%%%%%%%%%%%%%%%%%%%%%%%%%%%
%\theoremstyle{theorem}
\newtheorem{thm}{Theorem}[section]
\newtheorem{prop}{Proposition}[section]
\newtheorem{clly}{Corollary}[section]
\newtheorem{lem}{Lemma}[section]
\newtheorem{pf}{Proof}[section]
%\theoremstyle{definition}
\newtheorem{defi}{Definition}[section]
\newtheorem{exam}{Example}[section]
\newtheorem{rk}{Remark}[section]
\def\proof{\mbox {\it Proof.~}}
\newcommand{\en}{\~n}
\newcommand{\figura}{\stepcounter{figure}}
\newcommand{\cuadro}{\stepcounter{table}}
\renewcommand{\tablename}{Tabla}
\newtheorem{pot}{Proof of Theorem} %\ref{mainthm}}
%%%%%%%%%%%%%%%%%%%%%%%%%%%%%%%%%%%%%%%%%%
\def\N{\mathbb{N}}
\def\Z{\mathbb{Z}}
\def\Q{\mathbb{Q}}
\def\R{\mathbb{R}}
\def\C{\mathbb{C}}
\def\K{\mathbb{K}}
\def\V{\mathbb{V}}
\def\U{\mathbb{U}}
\def\O{\mathcal{O}}
\def\A{\mathcal{A}}
\def\L{\mathcal{L}}
\def\Rc{\mathcal{R}}
\newcommand{\vect}{\overrightarrow}
\newcommand{\modN}{\;\text{(mod $N$)}}

\usepackage[pass,paperwidth=15cm,paperheight=22cm]{geometry}
\paperheight=22cm
  \paperwidth=15cm
\setlength{\oddsidemargin}{-0.5cm} \setlength{\evensidemargin}{-1.0cm}
\setlength{\topmargin}{-1.5cm} \setlength{\textwidth}{11.5cm}
\setlength{\textheight}{17.5cm}


\setcounter{page}{1}
%%%%%%%%%%%%%%%%%%%%%%%%%%%%%%%%%%

\newcommand{\reffigure}[1]{Figura \ref{#1}}
\newcommand{\refpfigure}[1]{(\reffigure{#1})}

\newcommand{\reftable}[1]{Cuadro \ref{#1}}
\newcommand{\refptable}[1]{(\reftable{#1})}

\newcommand{\refalgorithm}[1]{Algoritmo \ref{#1}}
\newcommand{\refpalgorithm}[1]{(\refalgorithm{#1})}

\newcommand{\quotes}[1]{``#1''}

\newcommand{\sourcecode}[2][\footnotesize]{{\ttfamily#1#2}}

\newcommand{\link}[1]{{\footnotesize\url{#1}}}

%%%%%%%%%%%%%%%%%%%%%%%%%%%%%%%%%%%%%%%
\usepackage{fancyhdr}
\pagestyle{fancy}
%%%%%%%%%%%%%%%%%%%%%%%%%%%%%%%%%%%%%%%%%%%%%%%%%%%%%%%%%%%%%%%%%%%%%%%%%%%%%%%%%%%%%%%%%%%%%%%%%%%%%%%%%%%%%%%
%%%%%%%%SECCION DEL TITULO Y TIRILLA BIBLIOGRAFICA(ESTA ULTIMA PARA USO INTERNO DEL CUERPO EDITORIAL)%%%%%%%%%%
%%%%%%%%%%%%%%%%%%%%%%%%%%%%%%%%%%%%%%%%%%%%%%%%%%%%%%%%%%%%%%%%%%%%%%%%%%%%%%%%%%%%%%%%%%%%%%%%%%%%%%%%%%%%%%%
\begin{document}
\title{
\vspace{-1.1in}
\begin{flushleft}
{\normalsize \begin{center}
%{\em\bf Publicaciones en Ciencias y Tecnolog\'ia}\\ 
%\scriptsize{Vol 7, N$^{0}$2, Jul--Dic 2013, pp.7--15,   ISSN:1856-8890, Dep\'osito Legal:pp200702LA2730 }
%\small{Art\'iculo en revisi\'on.  ISSN:1856-8890, Dep\'osito Legal: pp200702LA2730}
\end{center}}
\end{flushleft}
\hrule \vspace{0.5in}{ARQUITECTURA HÍBRIDA DE NAVEGACIÓN PARA ROBOT PIONEER PD3X} }
%%%%%%%%%%%%%%%%%%%%%%%%%%%%%%%%%%%%%%%%%%%%%%%%%%%%%%%%%%%%%%%%%%%%%%%%%%%%%%%%%%%%%%%%%%%%%%%%%%%%%%%%%%%%%%%
%%%%%%%%SECCION AUTORES Y FILIACION DE LOS MISMOS %%%%%%%%%%%%%%%%%%%%%%%%%%%%%%%%%%%%%%%%%%%%%%%%%%%%%%%%%%%%%%
%%%%%%%%%%%%%%%%%%%%%%%%%%%%%%%%%%%%%%%%%%%%%%%%%%%%%%%%%%%%%%%%%%%%%%%%%%%%%%%%%%%%%%%%%%%%%%%%%%%%%%%%%%%%%%%

\author{ 
\footnote{  {\it\scriptsize  Decanato de Ciencias y Tecnolog\'ia,} 
{\it\scriptsize Universidad Centroccidental Lisandro Alvarado,}
{\it\scriptsize Barquisimeto, Venezuela, sauljabin@gmail.com}
}\hspace{1mm}{Saúl Piña}
 \hspace{3mm} 
\footnote{  {\it\scriptsize  Decanato de Ciencias y Tecnolog\'ia,} 
{\it\scriptsize Universidad Centroccidental Lisandro Alvarado,}
{\it\scriptsize Barquisimeto, Venezuela, thejorgemylio@gmail.com}
}\hspace{1mm}{ Jorge Parra} \\
 %\hspace{3mm} 
%\footnote{  {\it\scriptsize Departamento-Instituto-Facultad}, 
%{\it\scriptsize Universidad-Instituci\'on (filiaci\'on)},
%{\it\scriptsize Lugar, Pa\'is, email}
%}\hspace{5mm}{ Nombre autor 3 }
% \hspace{3mm} 
%\hspace{3mm}
%\footnote{ {\it\scriptsize  Departamento de Ingenier\'ia Industrial,  Instituto Tecnol\'ogico de Celaya}, 
%{\it\scriptsize Instituto Tecnol\'ogico de Celaya},  
%{\it\scriptsize M\'exico, gesparza@itc.mx }
%}\hspace{5mm} { Luis Gerardo Esparza-D\'iaz}
}

%\vspace{-15mm}
\date{\small{Octubre 2014}
}

%\date{}
%%%%%%%%%%%%%%%%%%%%%%%%%%%%%%%%%%
\maketitle 
%%%%%%%%%%%%%%%%%%%%%%%%%%%%%%%%%%%%%%%%%%%%%%%%%%%%%%%%%%%%%%%%%%%%%%%%%%
% SECCION TITULO CORTO E INICIALES AUTORES, HASTA DOS AUTORES:
% APELLIDO 1, INICIAL NOMBRE 1. ; APELLIDO 2, INICIAL NOMBRE 2.
%                        MAS DE DOS 
%                APELLIDO, INICIAL NOMBRE. et al
%%%%%%%%%%%%%%%%%%%%%%%%%%%%%%%%%%%%%%%%%%%%%%%%%%%%%%%%%%%%%%%%%%%%%%%%%%%
\fancyhead{} \fancyhead[CE]{{\scriptsize ARQUITECTURA HÍBRIDA DE NAVEGACIÓN PARA ROBOT PIONEER PD3X}}
\fancyhead[CO]{ Saúl Piña, Jorge Parra 
}
\fancyfoot{}
%\fancyfoot[CO,CE]{\scriptsize{{Publicaciones en Ciencias y Tecnolog\'ia.} Vol 7,N$^{0}$2, Jul--Dic 2013, pp.07--15.}}
%\fancyfoot[CO,CE]{Publicaciones en Ciencias y Tecnolog\'ia. Art\'iculo en revisi\'on}
\fancyfoot[RO,LE]{\thepage}

%%%%%%%%%%%%%%%%%%%%%%%%%%%%%%%%%%%%%%%%%%%%%%%%%%%%%%%%%%%%%%%%%%%%%%%%%%%%%
%%%%%%%%%%%%%%%%%%SECCION DE RESUMEN Y ABSTRACT%%%%%%%%%%%%%%%%%%%%%%%%%%%%%%
%%%%%%%%%%%%%%%%%%%%%%%%%%%%%%%%%%%%%%%%%%%%%%%%%%%%%%%%%%%%%%%%%%%%%%%%%%%%%
\begin{center}
{\bf\small Resumen}
%\vspace{-5mm}

\vspace{-3mm} \hspace{.05in}\parbox{4.5in} {{\small %\footnotesize 



 \textbf{Palabras clave}: Robótica, ARIA, AuRA, Arquitectura Híbrida, Pioneer P3DX.}}
\end{center}
\pagebreak


%\begin{center}
%{\bf\small T\'ITULO EN INGL\'ES \\ Abstract}
%
%\vspace{-3mm} \hspace{.05in}\parbox{4.5in} {{\small %\footnotesize 
%
%Cuerpo del abstract  
%
%     \textbf{Keywords}:  kw 1, kw 2, ... .}}
%\end{center}
%%%%%%%%%%%%%%%%%%%%%%%%%%%%%%%%%%%%%%%%%%%%%%%%%%%%%%%%%%%%%%%%%%%%%%%%%%%%%%%%%%%%%%%
%                           CUERPO DEL ARTICULO
%%%%%%%%%%%%%%%%%%%%%%%%%%%%%%%%%%%%%%%%%%%%%%%%%%%%%%%%%%%%%%%%%%%%%%%%%%%%%%%%%%%%%%%
%\pagebreak

\section*{Introducción}



\section*{Pioneer P3DX}

El Pionner P3DX \refpfigure{fig:pioneer} es un robot ideal para usar en el interior de las infraestructuras y cuenta con las siguientes dimensiones: 45,5 cm de largo por 38,1 cm de ancho por 23,7 cm de alto. El robot cuenta con 2 ruedas, las cuales se encuentran una a cada lado y cuenta cada una con un motor y un encoder. Además, tiene una rueda en la parte posterior que le ayuda a mantener el equilibrio.
Igualmente, el Pionner P3DX posee sensores de ultrasonido o sonares, los cuales sirven para detectar obstáculos en el ambiente.

\begin{figure}[here]
\begin{center}
\includegraphics[width=6cm]{pioneer.png} 
\caption{Robot Pioneer P3DX}
\label{fig:pioneer}
\end{center}
\end{figure} 

\subsubsection*{Especificaciones del Pioneer P3DX}
\\\\
\noindent\textbf{Construcción}
\\
\indent\textbf{Cuerpo:} 1,6 mm de aluminio.
\\
\indent\textbf{Neumáticos:} de caucho relleno de goma.
\\\\
\noindent\textbf{Operación}
\\
\indent\textbf{Peso del Robot:} 9 kg.
\\
\indent\textbf{Carga operativa:} 17 kg.
\\\\
\noindent\textbf{Potencia}
\\
\indent\textbf{Tiempo de ejecución:} 8-10 horas con 3 baterías.
\\
\indent\textbf{Tiempo de carga:} 12 horas o 2,4 horas.
\\
\indent\textbf{Baterías:} Soporta hasta 3 a la vez.
\\
\indent\textbf{Tensión:} 12 V.
\\
\indent\textbf{Capacidad:} 7,2 Ah.

\section*{Arquitectura AuRA}

La arquitectura AuRA \cite{arkin1997} (Autonomous Robot Architecture) propuesta por Ronald Arkin, comprende un enfoque híbrido
de navegación para robots móviles. Está constituida principalmente por dos capas, una deliberativa basada en la inteligencia artificial
tradicional y otra reactiva basada en teoría de esquemas. El la \reffigure{fig:aura} se puede observar la representación del AuRA, 
la capa deliberativa la componen el \textit{planificador de misión}, \textit{razonador espacial} y el \textit{secuenciador del plan}. En el nivel más alto se encuentra el \textit{planificador de misión}, representa la interfaz con el humano y permite establecer los objetivos, 
las restricciones y los parámetros en los que debe operar el robot. El \textit{razonador espacial} puede ser referido como navegador, este usa la información cartográfica
almacenada para construir la ruta deseada que el robot debe seguir para completar la misión. El \textit{secuenciador del plan}
puede ser comparado con un piloto, convierte la ruta generada por el \textit{razonador espacial} al conjunto de comportamientos (esquemas) que debe ejecutar la capa reactiva. En la capa reactiva el \textit{control de esquema}, es el responsable de llevar a cabo y monitorear en tiempo real el conjunto de comportamientos que el secuenciador ordena ejecutar, cada comportamiento esta asociado a un esquema perceptual, en otras palabras, cada estimulo genera una respuesta. 

\begin{figure}[here]
\begin{center}
\includegraphics[width=5cm]{aura.png} 
\caption{Representación de Arquitectura AuRA}
\label{fig:aura}
\end{center}
\end{figure} 

\subsubsection*{Fortalezas de la Arquitectura AuRA}

Modularidad, flexibilidad, generalización e hibridación, son los características resaltantes de esta arquitectura.

\textbf{Modularidad}, es altamente modular debido a su diseño en capas, lo que permite intercambiar o mejorar
los componentes sin afectar el funcionamiento general.

\textbf{Flexibilidad}, esta arquitectura permite integrar diferentes modelos de inteligencia artificial 
para la construcción de rutas, aprendizaje y adaptación. 

\textbf{Generalización}, puede ser empleado en distintos dominios como la manufactura, navegación, entre otros.

\textbf{Hibridación}, fusiona los paradigmas deliberativo y reactivo, lo que le permite al robot interactuar con eventos inesperados mientras intenta satisfacer sus metas.

\section*{Entorno de Desarrollo}

Para llevar a cabo la presente investigación se desarrollo una aplicación cliente
con las siguientes herramientas de software:

\textbf{ARIA} (Advanced Robot Interface for Applications) \cite{aria2014},
es una librería desarrollada en el lenguaje de programación C++ para todas las
plataformas MobileRobots/ActivMedia. Provee el conjunto de herramientas necesarias
para controlar y monitorear de forma dinámica al robot y sus elementos. Ademas, esta librería
incluye implementaciones llamadas \textit{wrapper} en los lenguajes de programación Python y Java, siendo este último 
el lenguaje de desarrollo para este investigación.

\textbf{MobileSim} \cite{mobilesim2014}, es un software para la simulación de plataformas
MobileRobots/ActivMedia. Permite usar la librería ARIA y sus wrappers. Es un ambiente
ideal para realizar experimentación y pruebas de los algoritmos desarrollados antes de
implementarlos en los robots reales.
 
\textbf{Eclipse IDE} \cite{eclipse2014}, es un entorno de desarrollo integrado para múltiples lenguajes de 
programación en los que destacan Java y C++. Provee de las herramientas necesarias para el desarrollo, depuración  
y compilación de aplicaciones.

\textbf{Java} \cite{java2014}, es un lenguaje de programación de propósito general, orientado a objetos y de alto nivel. 
 
\section*{Configuración del Entorno}

Antes de configurar el entorno de desarrollo es necesario descargar el código fuente del proyecto,
este está disponible al público con licencia de software \textit{GPL} en la siguiente dirección:
\\ \link{https://bitbucket.org/sauljabin/ariajava-p3dx}.

\subsubsection*{Configuración del Entorno Sobre Windows}

\begin{enumerate}
\item Previamente instalar Java y Eclipse.
\item Instalar \sourcecode{MobileSim-0.7.2-1.exe}, disponible en: \\ \link{http://robots.mobilerobots.com/wiki/MobileSim}.
\item Instalar \sourcecode{ARIA-2.7.6.exe}, disponible en: \\ \link{http://robots.mobilerobots.com/wiki/ARIA}.
\item Agregar a las variables de entorno del sistema la ruta: \\ \sourcecode{C:\textbackslash Program Files\textbackslash MobileRobots\textbackslash ARIA\textbackslash bin\textbackslash}.
\item Dirigirse a la ruta \sourcecode{C:\textbackslash Program Files\textbackslash MobileRobots\textbackslash ARIA\textbackslash bin\textbackslash}, cambiar el nombre del archivo
\sourcecode{AriaVC10.dll} a  \sourcecode{Aria.dll}.
\item Importar proyecto en eclipse.
\item En  \sourcecode{Java Build Path} se debe agregar el wrapper de Java que se encuentra en: \sourcecode{C:\textbackslash Program Files\textbackslash MobileRobots\textbackslash ARIA\textbackslash java\textbackslash Aria.jar}.
\item Verificar que la variable  \sourcecode{PATH\_LIBARIA} del archivo  \sourcecode{CONFIG.properties} 
	  sea igual a:  \sourcecode{C\textbackslash:\textbackslash\textbackslash Program Files\textbackslash\textbackslash MobileRobots\textbackslash\textbackslash ARIA\textbackslash\textbackslash bin\textbackslash\textbackslash }.
\end{enumerate}

\subsubsection*{Configuración del Entorno Sobre Debian/Ubuntu}

\begin{enumerate}
\item Previamente instalar Java y Eclipse. 
\item Descargar el archivo \sourcecode{MobileSim-0.7.3+gcc4.6.tgz}, disponible en: \\ \link{http://robots.mobilerobots.com/wiki/MobileSim}.
\item Ejecutar los siguientes comandos por consola para instalar MobileSim:
\sourcecode{ 
\\ \$ tar xzvf MobileSim-0.7.3+gcc4.6.tgz
\\ \# mv -f MobileSim-0.7.3 /usr/local/MobileSim
\\ \# ln -s -f /usr/local/MobileSim/MobileSim /usr/bin/mobilesim}
\item Descargar el archivo \sourcecode{ARIA-2.8.1+gcc4.6.tgz}, disponible en: \\ \link{http://robots.mobilerobots.com/wiki/ARIA}.
\item Ejecutar los siguientes comandos por consola para instalar ARIA:
\sourcecode{
\\ \$ tar xzvf ARIA-2.8.1+gcc4.6.tgz
\\ \# mv -f Aria-2.8.1 /usr/local/Aria
\\ \# echo '/usr/local/Aria/lib' > /etc/ld.so.conf.d/aria.conf
\\ \# ldconfig}
\item Importar proyecto en eclipse.
\item En  \sourcecode{Java Build Path} se debe agregar el wrapper de Java que se encuentra en: \sourcecode{/usr/local/Aria/java/Aria.jar}
\item Verificar que la variable \sourcecode{PATH\_LIBARIA} del archivo \sourcecode{CONFIG.properties} 
	  sea igual a: \sourcecode{/usr/local/Aria/lib/}
\end{enumerate}

\section*{Implementación de la Arquitectura}

% hablar de algoritmo de hormigas
% hablar de voronoi
% hablar de delauny

\section*{Herramienta de Software Desarrollada}

\section*{Experimentación}

\section*{Conclusión}

%%%%%%%%%%%%%%%%%%%%%%%%%%%%%%%%%%%%%%%%%%%%%%%%%%%%%%%%%%%%%%%%%%%%%%%%%%%%%%%%%%%%%%%%
%                                 BIBLIOGRAFIA
%%%%%%%%%%%%%%%%%%%%%%%%%%%%%%%%%%%%%%%%%%%%%%%%%%%%%%%%%%%%%%%%%%%%%%%%%%%%%%%%%%%%%%%%
\begin{thebibliography}{999}

\bibitem{arkin1997}
	Arkin, R. y Balch, T. (1997).
 	AuRA: Principles and practice in review.
	Journal of Experimental \& Theoretical Artificial Intelligence.
	Taylor \& Francis. Vol. 9. No. 2-3. Pp. 175-189.
	
\bibitem{aria2014} 
	MobileRobots. (2014).
	ARIA. 	
	Disponible en: \\\link{http://robots.mobilerobots.com/wiki/ARIA}.
	
\bibitem{mobilesim2014} 
	MobileRobots. (2014).
	MobileSim. 	
	Disponible en: \\\link{http://robots.mobilerobots.com/wiki/MobileSim}.
	
\bibitem{eclipse2014} 
	Eclipse. (2014).
	Eclipse IDE. 	
	Disponible en: \\\link{https://www.eclipse.org/ide/}.

\bibitem{java2014} 
	Oracle. (2014).
	Java. 	
	Disponible en: \\\link{http://www.oracle.com/technetwork/java/index.html}.

\end{thebibliography}
\end{document}
